\documentclass[10pt,landscape,a4paper]{article}


%*****************************************************************************
% Copyright (C) 2014 Volker Braun <vbraun.name@gmail.com>
% Creative Commons Attribution-Share Alike 3.0 License.
%*****************************************************************************


\usepackage{multicol}
\usepackage{calc}
\usepackage{ifthen}
\usepackage[landscape]{geometry}
\usepackage{amsmath,amsthm,amsfonts,amssymb}
\usepackage{color,graphicx,overpic}
\usepackage{hyperref}
\usepackage{inconsolata}
\usepackage[T1]{fontenc}
\usepackage{fancyvrb}
\usepackage[usenames,dvipsnames,svgnames,table]{xcolor}

\pdfinfo{
  /Title (git-cheat-sheet.pdf)
  /Creator (TeX)
  /Producer (pdfTeX 1.40.0)
  /Author (Volker Braun)
  /Subject (Example)
  /Keywords (pdflatex, latex,pdftex,tex)}

% This sets page margins to .5 inch if using letter paper, and to 1cm
% if using A4 paper. (This probably isn't strictly necessary.)
% If using another size paper, use default 1cm margins.
\ifthenelse{\lengthtest { \paperwidth = 11in}}
    { \geometry{top=.5in,left=.5in,right=.5in,bottom=.5in} }
    {\ifthenelse{ \lengthtest{ \paperwidth = 297mm}}
        {\geometry{top=1cm,left=1cm,right=1cm,bottom=1cm} }
        {\geometry{top=1cm,left=1cm,right=1cm,bottom=1cm} }
    }

% Turn off header and footer
\pagestyle{empty}

% Redefine section commands to use less space
\makeatletter
\renewcommand{\section}{\@startsection{section}{1}{0mm}%
                                {-1ex plus -.5ex minus -.2ex}%
                                {0.5ex plus .2ex}%x
                                {\normalfont\Large\bfseries}}
\renewcommand{\subsection}{\@startsection{subsection}{2}{0mm}%
                                {-1explus -.5ex minus -.2ex}%
                                {0.5ex plus .2ex}%
                                {\normalfont\normalsize\bfseries}}
\renewcommand{\subsubsection}{\@startsection{subsubsection}{3}{0mm}%
                                {-1ex plus -.5ex minus -.2ex}%
                                {1ex plus .2ex}%
                                {\normalfont\small\bfseries}}
\makeatother

% Define BibTeX command
\def\BibTeX{{\rm B\kern-.05em{\sc i\kern-.025em b}\kern-.08em
    T\kern-.1667em\lower.7ex\hbox{E}\kern-.125emX}}

% Don't print section numbers
\setcounter{secnumdepth}{0}


\setlength{\parindent}{0pt}
\setlength{\parskip}{0pt plus 0.5ex}

%My Environments
\newtheorem{example}[section]{Example}
% -----------------------------------------------------------------------

\begin{document}
\raggedright
\footnotesize
\begin{multicols}{3}


% multicol parameters
% These lengths are set only within the two main columns
%\setlength{\columnseprule}{0.25pt}
\setlength{\premulticols}{1pt}
\setlength{\postmulticols}{1pt}
\setlength{\multicolsep}{1pt}
\setlength{\columnsep}{2pt}

\newcommand{\note}[1]{\hfill\textrm{\textcolor{gray}{#1}}}
\newcommand{\args}[1]{\textit{\textcolor{blue}{#1}}}
\newcommand{\stdout}[1]{\textcolor{Sepia}{#1}}

% \fvset{frame=none,framesep=1mm,fontfamily=courier,fontsize=\scriptsize,numbers=left,framerule=.3mm,numbersep=1mm,commandchars=\\\{\}}

\fvset{gobble=2,framesep=1mm,commandchars=\\\{\},xleftmargin=2mm,xrightmargin=4mm}

\begin{center}
  \Huge\textbf{Sage, Git, \& Trac}
\end{center}

\section{Quickstart}

\subsection{Configuration}

You only need to do this once:
\begin{Verbatim}
  git config --global user.name "Your Name"
  git config --global user.email you@yourdomain.example.com
\end{Verbatim}
This data ends up in commits, so do it now before you forget!


\subsection{Get the Sage Source Code}

\begin{Verbatim}
  git clone git://github.com/sagemath/sage.git
\end{Verbatim}


\subsection{Branch Often}

A new branch is like an independent copy of the source code. Always
switch to a new branch \emph{before} editing anything:
\begin{Verbatim}
  git checkout \args{develop}\note{switch to the starting point}
  git branch \args{new\_branch\_name}\note{create new branch}
  git checkout \args{new\_branch\_name}\note{switch to new branch}
\end{Verbatim}
Without an argument, the list of branches is displayed:
\begin{Verbatim}
  git branch
  \stdout{  master}
  \stdout{* new_branch_name}\note{* marks the current branch}
\end{Verbatim}
When you are finished, delete unused branches:
\begin{Verbatim}
  git branch -d \args{branch\_to\_delete}
\end{Verbatim}


\subsection{Where Am I?}

Each change recorded by git is called a ``commit''. Examine history:
\begin{Verbatim}
  git show\note{show the most recent commit}
  git log\note{list in reverse chronological order}
\end{Verbatim}


\subsection{What Did I Do?}

This is probably the most important command. Example output:
\begin{Verbatim}
  git status
  \stdout{  On branch new_branch_name}\note{= current branch name}
  \stdout{Changes not staged for commit:}
  \stdout{  (use "git add <file>..." to update what will be committed)}
  \stdout{  (use "git checkout -- <file>..." to discard changes in}
  \stdout{   working directory)}
  \stdout{}
  \stdout{	modified:   modified_file.py}\note{= file you just edited}
  \stdout{}
  \stdout{Untracked files:}
  \stdout{  (use "git add <file>..." to include in what will be}
  \stdout{   committed)}
  \stdout{}
  \stdout{	new_file.py}\note{= file you just added}
  \stdout{}
  \stdout{no changes added to commit}
  \stdout{(use "git add" and/or "git commit -a")}
\end{Verbatim}


\subsection{Prepare to Commit}

When you are finished, tell git which changes you want to commit:
\begin{Verbatim}
  git add \args{filename}\note{add particular file}
  git add .\note{add all modified \& new}
\end{Verbatim}
The status command then lists the staged changes:
\begin{Verbatim}
  git st
  \stdout{On branch new_branch_name}
  \stdout{Changes to be committed:}
  \stdout{  (use "git reset HEAD <file>..." to unstage)}
  \stdout{}
  \stdout{	modified:   modified_file.txt}
  \stdout{	new file:   new_file.txt}
\end{Verbatim}


\subsection{Commit}

The commit command permanently records the staged changes. The new
commit becomes the new branch head:
\begin{Verbatim}
  git commit\note{opens editor for commit message}
  git commit -m "My Commit Message"
\end{Verbatim}
Commits cannot be changed, but they can be discarded and re-done with
the \texttt{--amend} switch. \emph{Never} amend commits that you have
already shared with somebody.

\section{Summary}

\textbf{workspace} is the file system: files that you can edit
\begin{Verbatim}
  git add \args{filename}\note{copy file to staging}
  git reset HEAD \args{filename}\note{copy staged file back}
\end{Verbatim}
\textbf{staging} is a special area inside the git repository
\begin{Verbatim}
  git commit\note{commit all staged files}
\end{Verbatim}
\textbf{commits} are the permanently recorded history
\begin{Verbatim}
  git checkout -- \args{filename}\note{copy file from repo to workspace}
\end{Verbatim}


\section{Merging}

A commit with more than one parent is a merge commit:
\begin{Verbatim}
  git merge \args{other\_branch}\note{incorporate other branch/commit}
\end{Verbatim}
If there is no conflict this automatically creates a new merge
commit. Otherwise, the conflicting regions are marked like this:
\begin{Verbatim}
  \stdout{Here are lines that are either unchanged from the common}
  \stdout{ancestor, or cleanly resolved because only one side changed.}
  \stdout{<<<<<<< yours:source_file.py}
  \stdout{Conflict resolution is hard;}
  \stdout{let's go shopping.}
  \stdout{=======}
  \stdout{Git makes conflict resolution easy.}
  \stdout{>>>>>>> theirs:source_file.py}
  \stdout{And here is another line that is cleanly resolved or unmodified.}
\end{Verbatim}
Edit as needed; To finish, run one of:
\begin{Verbatim}
  git commit\note{commit your merge conflict resolution}
  git merge --abort\note{discard merge attempt}
\end{Verbatim}


\section{Branch Heads}

A git branch is just a pointer to a commit. This commit is called the
branch \texttt{HEAD}. You can point it elsewhere with
(\texttt{--hard}) or without (\texttt{--soft}, less common) resetting
the actual files. That is, the following discards content of the
current branch and makes it indistinguishable from a new branch that
started at \verb!new_head_commit!:
\begin{Verbatim}
  git reset --hard \args{new_head_commit}
\end{Verbatim}

There are various ways to specify a commit to reset to:
\begin{Verbatim}
  3472a854df051b57d1cb7e4934913f17f1fef820\note{40-digit SHA1}
  3472a85\note{the first few digits of the SHA1}
  branch_name\note{the name of another branch pointing to it}
  6.2.beta6\note{a tag in the Sage git repo; Every version is tagged}
  origin/develop\note{the \texttt{develop} branch in the remote \texttt{origin}}
  HEAD~\note{first parent of the current head}
  HEAD~2\note{first parent of the first parent of the current head}
  HEAD^2\note{second parent of the current head}
  FETCH_HEAD\note{commit downloaded with the \texttt{git fetch} command}
\end{Verbatim}


\section{Trac and the Sage Git Repo}

At \url{http://git.sagemath.org} you can browse our own git
repository. On trac tickets, you can click on the links under
\textbf{Branch:}

\subsection{Git Trac Subcommand}

We have added a \texttt{git trac} command to interact with our git and
trac server. You can download and temporarily enable it via
\begin{Verbatim}
  git clone git@github.com:sagemath/git-trac-command.git
  source git-trac-command/enable.sh
\end{Verbatim}
See the developer guide for how to install it on your system.

\subsection{Configure Git Trac}

To make changes to trac you need to have an account:
\begin{Verbatim}
  git trac config --user \textcolor{blue}{USER} --pass \textcolor{blue}{PASS}
\end{Verbatim}
Furthermore, our git repository uses your SSH keys for
authentication. Log in on \url{https://trac.sagemath.org} and go to
Preferences $\to$ SSH keys.

\subsection{Downloading / Creating a Branch}

\begin{Verbatim}
  git trac checkout \args{ticket_number}\note{branch for existing ticket}
  git trac create \args{"Ticket Title"}\note{create new ticket}
\end{Verbatim}
This will get the branch from trac, or create a new one if there is
none yet attached to the ticket.

\subsection{Pull Changes from Trac}

\begin{Verbatim}
  git trac pull \args{optional_ticket_number}
\end{Verbatim}
The trac ticket number will be guessed from a number embedded in the
current branch name, or if there is a branch of the same name on a
ticket already. 


\subsection{Push your Changes to Trac}

\begin{Verbatim}
  git trac push \args{optional_ticket_number}
\end{Verbatim}


\section{Getting Help}

\begin{Verbatim}
  git help \args{command}\note{show help for (optional) command}
  git trac create -h\note{help for subcommand}
\end{Verbatim}
Sage developer guide: \url{http://www.sagemath.org/doc/developer/}




% % You can even have references
% \rule{0.3\linewidth}{0.25pt}
% \scriptsize
% \bibliographystyle{abstract}
% \bibliography{refFile}

\end{multicols}
\end{document}





